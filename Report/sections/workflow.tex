% !TEX root = ../report.tex

\section{Work flow}
The workflow of our project can be divided in different phases, from the idea selection to the testing of the application. Our goal was to create something which would have made the experience at the EXPO both more interesting and educational. The growing technology related to the Augmented Reality were perfect for stimulating the visitors interest while the creation of the dishes from all over the world would have made the experience more challenging and instructive. The Augmented Reality feature is a core component in our application so the choice of the correct library was very important and the Metaio SDK has revealed to be a good decision because it provides all the functionality needed by us without having to buy the licence. Another important issue related to the Augmented Reality functionality was the creation of the 3D model representing the mascots. We aren't expert in 3D model creation, for this reason we tried to search already existing models and edit them in order to make them look more like nice mascots. The software which was very useful in this phase was Blender, an open source 3D graphics and animation software. Another important feature of our application is the creation of recipes by combining different ingredients; for this reason we needed to define in a precise way the set of ingredients, recipes and mascots we would insert in our application.
The most important phase from the development point of view was the structural analysis of the entire system. During this stage we have defined the classes organization of the Android application and we have discussed about the usefulness of a web application. At the end we come up with the idea that the web application not only can be used to provide dynamic content to the application but can also allow us to create and manage the database of receipts and ingredients. Using the documents redacted in the structural analysis phase we have implemented the web application based on the ExpressJS framework and using a MongoDB backend database. The data inside the web service are exported to a SQLite database in order to be used in the application. Finally the last two phases consist of implementing the Android application and testing in order to solve unexpected bugs.